\newpage
\section{ВИСНОВКИ}
\subsection{Наукова новизна отриманих результатів}
Враховуючи результати експерементів було знайдено ключові фактори впливу на якість та ефективність роботи моделей  глибокого навчання у системах рекомендацій.
Розроблений алгоритм дає можливість прискорити прототипування, перевірку і порівняння існуючих та майбутніх моделей систем рекомендацій.
\subsection{Практичне значення отриманих результатів}
Було сформульовано і змодельовано наступні фактори впливу 
\begin{enumerate}
    \item Вибір спліт методу. Методу  Dataset Spliting Method
    \item Вибір метрики якості Evaluation Metric Selection
    \item Формулювання функції втрат для оптимізації Loss function Design
    \item Мотодологія негативного семплінгу. Негативний Negative Samoling Strategy
    \item Стратегія ініціалізації вагів. Parap Init Strategy
    \item Вибір алгоритму оптимізації Model Optimizer Selection 
    \item Вибір методу регуляризації Regularization Term /  Dropout
    \item Критерії останова Early Stop Mechanism
    \item Донавчання вагів Hyper Params Tuning
    \end{enumerate}
Проведене експерементальне дослідження їх впливу на якість роботи  обраних алгоритмів рекомендацій. 
Розроблене ПЗ пришвидшує відбір оптимальних моделей і їх якісне порівняння. ПЗ може використовуватись як і для комерційних цілей так і (наукових) для порівняння нових алгоритмів.

У даній роботі було проаналізовано задачу побудови рекомендацій, розглянуто її проблематику. Розглянуто широкий спектр метрик якості із використанням підходу до класифікації, ранжування і різноманіття.
На основі алгоритмів нейромережевої факторизації, варіаційного автоенкодера і графової нейронної колаборативної фільтрації було розглянуто підхід глибокого навчання до побудови рекомендацій.
Сформовано перелік факторів впливу на метрики якості. Проведений аналіз їх впливу і можливі рішення для усунення.
На основі відкритих наборів даних було імплементовано розглянуті алгоритми рекомендації і проведене їх порівняння.
