\newpage
\section{ОГЛЯД СУЧАСНОГО СТАНУ ОБЛАСТІ ПОБУДОВИ СИСТЕМ РЕКОМЕНДАЦІЙ}
У наслідок зривного росту сфери інформаційних технологій і стрімким пересіченням буденного життя звичайних людей із системами обміну данними відкриваються нові можливості для впливу на процеси прийняття рішень, таких як - який фільм переглянути, прослухати музику або яку річ купити. Крім того за останнє десятиліття користувачі стикаються із дилемою вибору серед великої кількості варіантів. Як результат, автоматичні рекомендації є важливими для покращення досвіду користувачів та зменшення перевантаження інформації. Загалом, системи рекомендацій відіграли незамінну роль у різних системах фільтрації інформації для підвищення цінності бізнесу та полегшення процесів прийняття рішень. Netfilx - типовий приклад системи рекомендацій (Рис. 1.1).

\begin{figure}[H]
    \centering
    \includegraphics[width=1\textwidth]{images/netflixl.png}
    \caption{Інтерфейс головної сторінки стрімінгової платформи Netflix. }

\end{figure}

У своїй найпростішій формі персоналізовані рекомендації пропонуються як ранжовані списки елементів. Виконуючи це рейтингування, RS намагаються передбачити які продукти чи послуги є найбільш релевантними, виходячи з уподобань і обмежень користувача. Щоб виконати таке обчислювальне завдання, RS збирають інформацію від користувачів щодо їхніх уподобань, які або явно виражені, наприклад, як рейтинги для продуктів, або виводяться шляхом інтерпретації дій користувача. Наприклад, RS може розглядати навігацію до сторінки конкретного продукту як неявний знак переваги для елементів, показаних на цій сторінці.

В останні роки інтерес до рекомендаційних систем різко зріс, про це свідчать такі факти:
\begin{enumerate}
    \item Системи рекомендацій відіграють важливу роль на високорейтингових інтернет-сайтах, таких як Amazon.com, YouTube, Netflix, Spotify, LinkedIn, Facebook, Tripadvisor, та IMDb. Крім того, багато медіа-компаній зараз розробляють і розгортають RSs як частину послуг які вони надають своїм передплатникам. Наприклад, Netflix, онлайн-провайдер потокового мультимедіа, присудив приз у мільйон доларів команді, яка першою зуміла суттєво покращити продуктивність своєї системи рекомендацій.

    \item Активна робота введеться у наукові сфері. Популярні академічні журнали активно публікують результати досліджень у сфері систем рекомендацій. Існують різні методи проектування цих систем, починаючи від простого (наприклад, заснованого лише на номінальних елементах від одного користувача) до надзвичайно складного. Складні рекомендаційні системи використовують різноманітні джерела даних і часто використовують нелінійні методи навчання. Таким чином, завдання рекомендації забезпечує відмінний простір для застосування методологій машинного навчання. Оскільки користувачі продовжують споживати вміст і дають більше даних, ми можемо створити системи на основі машинного навчання для використання цих даних для надання кращих та кращих рекомендацій.
\end{enumerate}

\subsection{Призначення системи рекомендацій}
Система рекомендацій знайшла своє призначення у багатьох сферах виконуючи наступні завдання:
\begin{itemize}
    \item Збільшення кількості проданого товару. З точки зору бізнесу, збільшення конвертацій (співвідношення кількості переглядів до цільових дій - покупки, продажу, перегляду і т.д.) є головною метою.
    \item Збільшення різноманітності (diversity) проданого товару. Тобто реалізація непопулярного товару без ризику втратити прихильність користувачів
    \item Збільшення задоволеності користувачів. Добре розроблена система рекомендацій може покращити досвід користувачів у програмі. Користувач вважатиме рекомендації цікавими, актуальними та приємними. Це призводить до більш високого загального задоволення.
    \item Зрозуміти потреби користувача. Хороша система рекомендацій детально описує бажання користувачів, явно або неявно. Тоді бізнес може вирішити повторно використовувати ці знання для багатьох інших цілей.
\end{itemize}
\subsection{Прикладні сфери використання}

Дослідження рекомендаційних систем, як правило, проводяться з сильним акцентом на практичні та комерційні програми. Домен застосування суттєво впливає на тип алгоритмічного підходу, який слід застосовувати. Умовно можна надати таку таксономію систем рекомендацій що класифікує існуючі програми до конкретних доменів:
\begin{itemize}
    \item  Розваги - Рекомендації щодо фільмів (Netflix), Музика (Spotify, Pandora) та мобільних додатків (Apple Store, Android Store).
    \item Соціальні мережі.
    \item Вміст-персоналізовані газети (The New York Times, The Wall Street Journal), рекомендація щодо зображень (Pinterest), рекомендації веб-сторінок, електронного навчання (Coursera) та електронних листів (Gmail).
    \item Електронна комерція - Рекомендації для споживачів продуктів для придбання, таких як книги (Amazon), косметика, одяг та багато іншого.
    \item Рекомендації туристичних послуг (Skyscanner), експертів з консультацій (Styleseat, ClassPass), будинки для оренди (AirBnB) або послуги із пошуку партнера (Tinder, Bumble).
\end{itemize}

\subsection{Проблематика напрямку}
Хоча завдання створення автоматичної рекомендації не є новим, залишається багато проблем які сприяють дослідженням сучасних систем, особливо у задачах колаборативної фільтрації, які мають на меті змоделювати поведінку користувачів для полегшення рекомендацій. Основі питання:
\begin{itemize}
    \item Розрідженість (Data sparsity): Проблема розрідження є досить помітною у системах колаборативної фільтрації, оскільки лише невелика кількість користувачів надає рейтинги, і лише невелика кількість предметів має достатню кількість оцінок. Це явище відоме як "проблема холодного старту" (сold start problem), що запобігає системі рекомендацій генерувати змістовні прогнози для нових користувачів через обмеженість даних.
    \item Надмірна спеціалізованісь (Over specialized): Система рекомендує об’єкти дуже схожі до тих які вже відомі користувачу. Незважаючи на те, що це свідчить про хороший загальний рівень прогнозувань моделі (її ефективність), система не привносить різноманіття, відкриття нового, не знайомого контенту.
    \item Упередження на основі популярності (Popularity bias | "Harry Potter Effect"): Частково проблема систем колаборативної фільтрації. Через нерівномірність матриці взаємодій алгоритм рекомендує об’єкти із великою кількістю оцінок (відгуків), а при  недостатній кількості обє’кти ігноруються.
    \item Точність: Системи рекомендацій повинні забезпечити високий рівень точності прогнозування, щоб забезпечити якість, яку користувачі часто вимагають. У дослідженні точність зазвичай оцінюється/досліджується шляхом прогнозування рейтингу та рейтингу предметів.
    \item Маштабованість: Системи рекомендацій у реальному світі розгортаються в динамічному і інтерактивному середовищі - дані користувачів та елементів швидко надходять і виходять. Важливо вирішити обчислювальні вимоги та час навчання/виведення, необхідні для ефективної обробки цих даних.
    \item Шахрайство.
          У рекомендаційних системах, де кожен може ставити оцінки, люди можуть давати позитивні оцінки своїм предметам і погані своїм конкурентам. Також, рекомендаційні системи стали сильно впливати на продажі та прибуток, з тих пір як отримали широке застосування в комерційних сайтах. Це призводить до того, що недобросовісні постачальники намагаються шахрайським чином піднімати рейтинг своїх продуктів і знижувати рейтинг свої конкурентів.

    \item Різноманітність.
          Колаборативна фільтрація спочатку визнана збільшити різноманітність, щоб дозволяти відкривати користувачам нові продукти з незліченної множини. Однак деякі алгоритми, зокрема основні на продажах і рейтингах, створюють дуже складні умови для просування нових і маловідомих продуктів, так як їх заміщають популярні продукти, які давно перебувають на ринку. Це в свою чергу тільки збільшує ефект «багаті стають ще багатшими» і приводить до меншої різноманітності.

    \item Білі ворони.
          До «білих ворон» відносяться користувачі, чия думка постійно не збігається з більшістю інших. Через унікальність смаку їм неможливо щось рекомендувати. Однак, такі люди мають проблеми з отриманням рекомендацій і в реальному житті, тому пошуки вирішення даної проблеми в даний час не ведуться.

    \item Проблема холодного старту.
          Нові предмети або користувачі представляють велику проблему для рекомендаційних систем. Частково проблему допомагає вирішити підхід, заснований на аналізі вмісту, так як він покладається не на оцінки, а на атрибути, що допомагає включати нові предмети в рекомендації для користувачів. Однак проблему з наданням рекомендації для нового користувача вирішити складніше.

    \item Синонімія.
          Синонімією називається тенденція схожих і однакових предметів мати різні імена. Більшість рекомендаційних систем не здатні виявити ці приховані зв'язки і тому відносяться до цих предметів як до різних. Наприклад, «фільми для дітей» та «дитячий фільм» відносяться до одного жанру, але система сприймає їх як різні.

    \item Розрідженість даних.
          Як правило, більшість комерційних рекомендаційних систем заснована на великій кількості даних (товарів), в той час як більшість користувачів не ставить оцінки товарам. В результаті цього матриця «предмет-користувач» виходить дуже великою і розрідженою, що представляє проблеми при обчисленні рекомендацій. Ця проблема особливо гостра для нових, щойно створених систем. Також розрідженість даних підсилює проблему холодного старту.
\end{itemize}

\subsection{Класифікація систем рекомендацій}
Існує два способи екстраполяції рейтингу від відомих об’єктів до невідомих.
Перший полягає у створені евристики, яка визначається функцією подібності та емпірично підтверджує ефективність. Друга навчається використовуючи функцію втрат, наприклад MSE або сross-entropy loss. Отримавши оцінку рейтингу для об’єктів рекомендацій ми рекомендуємо top n елементів із найвищим рейтингом.
Тому загальна класифікація систем рекомендацій наступна (Рис 1.2):
\begin{itemize}
    \item Фільтрування на основі контенту: система рекомендує нові елементи із характеристиками подібними до існуючих елементів, яким користувач віддав перевагу в минулому. Отже, генерація рекомендацій повністю залежить від історії взаємодій із елементами.
    \item Колаборативна фільтрація: система рекомендує об’єкти новим користувачам зі смаками схожими до існуючих користувачів у базі даних.Отже, якість рекомендацій залежить від уподобань користувача.
    \item Гібридний підхід: гібридна система поєднує фільтрацію на основі вмісту та колаборативну фільтрацію для створення рекомендацій.
\end{itemize}
\begin{figure}
    \centering
    \includegraphics[width=1\textwidth]{images/Rec_system_types.png}
    \caption{Поділ систем рекомендацій за категоріями}
\end{figure}
\subsubsection{Фільтрування на основі вмісту}
У системах рекомендацій на основі вмісту функція корисності
$F(u,i)$ елемента $i$ для користувача $u$ оцінюється на основі
функцій $f(u,i_{k})$, призначеного користувачем $u$ для кожного
елемента $i_{k} \in I$, схожий на предмет $i$. Наприклад, у системі
рекомендацій музики, щоб рекомендувати нові пісні користувачеві
$U$,підхід до рекомендацій на основі вмісту намагається зрозуміти
подібність між піснями, які користувач u часто слухав у минулому.
Тоді лише пісні, які мають високу ступінь подібності з будь-якими
уподобаннями користувача, згодом рекомендуються.
Основні обмеження систем рекомендацій на основі вмісту:
\begin{enumerate}
    \item Обмежений аналіз: підходи на основі контенту обмежені функціями, які явно пов'язані з рекомендованими елементами. Отже, щоб мати можливість дати оцінку схожості, вміст повинен бути:
          \begin{enumerate}
              \item У форматі, який можна обробити автоматично
              \item Або може бути призначений елементам вручну.
          \end{enumerate}
          У першому сценарії створитти функції оцінки схожості в неструктурованих даних непросто, такі як зображення та аудіо. У другому сценарії часто недоцільно призначати атрибути вручну через обмежені обчислювальні та людські ресурси.
    \item Гомогенність: Коли система може рекомендувати лише елементи, які високо оцінюють профіль користувача, користувач обмежується лише предметами, подібними до тих, що вже оцінені. Іншими словами, підходи, засновані на контенті, не дають різноманітних рекомендацій. В ідеалі користувачеві слід представити широкий спектр варіантів, а не просто однорідний набір альтернатив.
    \item Рекомендації для нового користувача: Користувач повинен оцінити достатню кількість предметів до того, як система рекомендацій на основі вмісту може зрозуміти його/її вподобання та представити його/їй довірливими предметами. Таким чином, новий користувач, який не має попередніх рейтингів, не отримає точних рекомендацій
\end{enumerate}
\subsubsection{Колаборативна фільтрація}

У системах колаборативної фільтрації ми намагаємось передбачити корисність елементів для певного користувача на основі елементів, раніше оцінених іншими користувачами. Тобто, функція корисності $F(u, i)$ елемента $I$ для користувача $U$ оцінюється на основі функцій $F(u_{j}
    , i)$, призначеного елементу $I$ тими користувачами $u_{j} \in U$, які схожі на користувача $u$. Наприклад, у контексті програми рекомендацій щодо книги, щоб рекомендувати книги користувачеві $U$, система спільної фільтрації спочатку знаходить «однодумців» користувача $U$ або інших користувачів із подібними смаками в книгах (які оцінюють однакові книги аналогічно). Тоді, рекомендується лише книги, які найбільше подобаються похожим користувачам.

Алгоритми колаборативної фільтрації можна розділити на два основні класи: на основі сусідства та на основі моделі.

\subsubsection{Підхід оснований на сусідстві}
Алгоритм, заснований на сусідстві, обчислює подібність двох користувачів або виробів, виробляє прогноз для користувача, приймаючи середнє зважене всіх оцінок. Обчислення схожості між виробами або користувачами є важливою частиною цього підходу. Багаторазові заходи, такі як кореляції Пірсона і схожість, заснована на скалярному добутку, використовується для цього.

Схожість двох користувачів X, Y через кореляцію Пірсона визначається як:
\[simm(x,y) = \frac{{}\sum_{i \in I_{xy}}(r_{x,i} - \overline{r}_{x})(r_{y,i} - \overline{r}_{y})}
    {\sqrt{\sum_{i \in I} (r_{x,i} - \overline{r}_{x})^2(r_{y,i} - \overline{r}_{y})^2}}\] - де  $I_{x,y}$ набір елементів оцінений як користувачем x так і користувачем y.

Заснований на користувачеві алгоритм top-N рекомендації використовує засновану на подібності векторну модель для визначення K — більшості подібних користувачів до активного користувача. Після того, як знайдені найбільш схожі користувачі, їх відповідні матриці агрегуються для визначення рекомендованого набору елементів. Популярний метод, знаходження схожих користувачів — Locality-sensitive hashing, який реалізує механізм пошуку найближчих сусідів у лінійному часі.

Переваги цього підходу включають в себе: очікуваність результатів, що є важливим аспектом рекомендаційних систем; просте створення і використання; просте полегшення нових даних; добра масштабованість зі співавторами рейтингових пунктів.

Є також кілька недоліків при такому підході. Його продуктивність знижується, коли дані становляться розрідженими, що трапляється часто з виробами, пов'язаними з мережею. Це ускладнює масштабованість такого підходу і створює проблеми з великими наборами даних. Хоча він може ефективно обробляти нових користувачів, тому що спирається на структури даних, додавання нових елементів стає більш складним, що, як правило, спирається уявленням про конкретну складову векторного простору. Додавання нових елементів вимагає включення нового пункту і повторного включення всіх елементів у структурі.

\subsubsection{Підхід заснований на моделі}

Даний підхід надає рекомендації, вимірюючи параметри статистичних моделей для оцінок користувачів, побудованих за допомогою таких методів як, метод баєсовских мереж, кластеризації, латентно-семантичної моделі , такі як сингулярний розклад, імовірнісний латентно-семантичний аналіз, прихований розподіл Дирихле і марковський процес вирішування на основі моделей. Моделі розробляються з використанням інтелектуального аналізу даних, алгоритмів машинного навчання, щоб знайти закономірності на основі навчальних даних. Число параметрів в моделі може бути зменшено в залежності від типу за допомогою методу головних компонент.

Цей підхід є більш комплексним і дає більш точні прогнози, оскільки допомагає розкрити латентні фактори, що пояснюють спостережувані оцінки.

Даний підхід має ряд переваг. Він обробляє розріджені матриці краще, ніж підхід заснований на сусідстві, що в свою чергу допомагає з масштабністю великих наборів даних.

Недоліки цього підходу полягають в «дорогому» створенні моделі. Необхідний компроміс між точністю і розміром моделі, тому що можна втратити корисну інформацію у зв'язку із скороченням моделей.

Алгоритми на основі пам'яті-це евристика, яка прогнозує рейтинги на основі всієї колекції раніше оцінених елементів користувачами системи.
Алгоритми на основі моделі використовують колекцію рейтингів для навчання моделі (ML), які після валідації потім використовуються для прогнозування рейтингів.
% Колаборативні системи також мають особливі обмеження:
% \begin{enumerate}
%     \item Cold Start User: Це та сама проблема, що і у підходів, що базуються на Contentbaste. Щоб зробити точні рекомендації, система повинна спочатку вивчити переваги користувача з попередніх рейтингів.
%     \item Cold Start Item: Оскільки підходи спільної фільтрації покладаються виключно на переваги користувачів для надання рекомендацій, новий елемент не буде рекомендовано, поки він не буде оцінений достатньою кількістю користувачів.
%     \item Обмеженність (Data Sparsity): У будь -якій системі кількість вже отриманих оцінок зазвичай невелика порівняно з кількістю рейтингів, які потрібно прогнозувати. Ефективне прогнозування оцінок з невеликої кількості прикладів є досить важливим. Крім того, успіх спільної фільтрації залежить від наявності критичної маси користувачів. Наприклад, може бути багато продуктів, придбаних лише кількома людьми в системі рекомендацій щодо продуктів, а це означає, що вони були б рекомендовані дуже рідко, навіть якщо ці кілька користувачів дали ці товари високі рейтинги. Крім того, для користувача, чиї смаки незвичні порівняно з рештою населення, не буде жодних інших суттєво подібних користувачів, що дає погані рекомендації.
% \end{enumerate}

\subsubsection{Гібридні системи}
Даний підхід об'єднує в собі підхід заснований на сусідстві і заснований на моделі. Гібридний підхід є найпоширенішим при розробці рекомендаційних систем для комерційних сайтів, так як він допомагає подолати обмеження початкового оригінального підходу (заснованого на сусідстві) і поліпшити якість прогнозів. Цей підхід також дозволяє подолати проблему розрідженості даних і втрати інформації. Однак даний підхід складний і дорогий у реалізації та застосуванні.

\subsection{Постановка задачі системи рекомендацій}

У рекомендаційних системах релевантність об’єкта зазвичай представлена рейтингом, що вказує наскільки конкретному користувачеві подобається певний елемент. Корисність може бути довільною функцією і залежати від завдання.
Наприклад, елемент буде кориснішим, якщо він збільшує задоволення користувачів або краще підкріплює потреби користувача. Залежно від постановки, оцінка F може бути вказана користувачем - explicit feedback(як це часто робиться в контексті заповнених користувачами рейтингів), або обчислюється програмно - implicit feedback (значення бінаризовані, приймають 0 або 1).

Зважаючи на релевантність, строго, проблема рекомендації може бути сформульована таким чином:
\begin{itemize}
    \item Нехай $U$ будете набором всіх користувачів і нехай $I$ буде набором усіх елементів. Обидва ці простори можуть бути дуже великими - аж до потенційно мільйонів предметів.
    \item Нехай $F$ - функція важливості (корисності), яка вимірює відповідність елемента $i$ до користувача $u$ наступним чином: $f: U \times I \rightarrow R$,  де $R$ - це упорядкований набір налаштувань користувачів для елементів.
    \item Для кожного користувача $u \in U$ ми хочемо вибрати елемент $i \in I$, який максимально збільшує задоволеність користувача.
\end{itemize}

Тобто, ми хочимо вирішити наступну задачу оптимізації:
\[\forall u \in U, i_{s} = arg max_{i \in I}F(u,i)\]
де кожен елемент із простору користувачів U можна визначити за допомогою профілю, який включає різні характеристики, такі як ідентифікатор користувача, вік, стать, дохід тощо.
Аналогічно, кожен елемент простору елемента I визначається за допомогою (зокрема) набору характеристик. Наприклад, у завданнях музичної рекомендації -  Spotify, SoundCloud, Pandora, де I є колекцією пісень, кожна пісня може бути представлена не лише за її ідентифікатором, але і за назвою, жанром, виконавцем, роком випуску тощо.

Центральна проблема, з якою стикається будь-яка система рекомендацій, полягає в тому, що функція F зазвичай не визначається у всьому просторі $U \times I$ - вона, в кращому випадку, визначається лише на підмножині цього простору. Це означає, що F потрібно екстраполювати на весь простір $U \times I$.
У системах рекомендацій, функція F зазвичай представлена рейтингом і спочатку визначається лише на елементах, які раніше оцінювали (існуючі) користувачі. Наприклад, у програмі рекомендацій книг, як PlayBooks, користувачі спочатку оцінюють деяку підмножину книг, які вони вже читали. Мета програми - передбачити рейтинг до книг які не відомі користувачу, та відобразити відповідні рекомендації на основі цих прогнозів.

\subsection*{Висновок}
У розділі розглянуто постановку задачі системи рекомендацій. Наведено типові сфери використання, користь бізнесу і цілі. На основі розглянутої літератури було розглянуто класифікацію систем відносно підходу до побудови рекомендацій.
Проаналізовано проблематику напрямку.